\documentclass[12pt,a4paper]{scrartcl}
\usepackage[utf8]{inputenc}
\usepackage[english,russian]{babel}
\usepackage{indentfirst}
\usepackage{misccorr}
\usepackage{graphicx}
\usepackage{amsmath}
\begin{document}
\begin{titlepage}
  \begin{center}
    \large
 
    УНИВЕРСИТЕТ
    \vspace{0.25cm}
     
    Факультет 
     
    Кафедра 
    \vfill
 
    \textsc{Отчет по лабораторной работе №1}\\[5mm]
     
    {\LARGE Работа с сервоприводом}

\end{center}
\vfill
 
\newlength{\ML}
\settowidth{\ML}{«\underline{\hspace{0.7cm}}» \underline{\hspace{2cm}}}
\hfill\begin{minipage}{0.4\textwidth}
  Руководитель: \\
  Выполнила:  \\
  
  
\end{minipage}%
\bigskip

 
\begin{center}
  Санкт-Петербург
\end{center}
\end{titlepage}

\begin{tableofcontents}
\end{tableofcontents}
    \newpage
    
\section{Введение}

\label{sec:intro}
 
% Что должно быть во введении

\subsection{Описание работы}
Провести лабораторную работу по работе с сервоприводом при помощи микроконтроллера Arduino Uno и программы Arduino IDE.\\
\\

\subsection{Цель работы}
Научиться работать с сервоприводом, управляя его работой с помощью микроконтроллера, решить сопутствующие проблемы, понять общий принцип работы с датчиками при помощи Arduino Uno и Arduino IDE.

 	\newpage
 
\section{Оборудование, использованное в работе}
\begin{figure}[h!]  
	\centering
	\includegraphics[width=0.5\textwidth]{pic1.eps} %тут в конце адрес картинки 
	\caption{Микроконтроллер Arduino Uno} 
	\label{image:pic1} %% внутренняя ссылка на картинку в рамках документа 
\end{figure}	
\begin{figure}[h!]  
	\centering
	\includegraphics[width=0.5\textwidth]{pic2.eps} %тут в конце адрес картинки 
	\caption{Сервопривод} 
	\label{image:pic2} %% внутренняя ссылка на картинку в рамках документа 
\end{figure}
  \newpage
  \section{Ход работы}
  \subsection{Схема подключения}
  \begin{figure}[h!]  
  	\centering
  	\includegraphics[width=0.5\textwidth]{pic3.eps} %тут в конце адрес картинки 
  	\caption{Схема подключения Сервопривода к Ардуино} 
  	\label{image:pic3}
  \end{figure}
  \subsection{Код работы}
 \begin{verbatim}
byte x =0;
void setup () {
// put your setup code here , to run once :
pinMode (12 , OUTPUT ) ;
Serial . begin (9600) ;
}
void loop () {
analogWrite (12 ,0) ;
delay (600) ;
x =0;
while (x <255) {
Serial . println ( x ) ;
x = x +15;
analogWrite (12 , x ) ;
delay (1000) ;
}
}

 \end{verbatim}
  \newpage
\section{Результат работы}
\subsection{Таблица}

\begin{tabular}{|p{8cm}|p{8cm}|}
	\hline
\centering	S &  Поведение Сервопривода \\ \hline
0 & Исходное положение \\ \hline
+15 (х13) & Вал стабильно поворачивается на одинаковые углы по часовой стрелке при подаче одинаковых пределенных импульсов (15) \\ \hline
+15 & Вал поворачивается в обратную сторону \\ \hline
+15 & Сервопривод игнорирует сигналы, после чего возвращается в исходное положение  \\ \hline

\end{tabular}
\subsection{Принцип работы}
Приводы подключаются к программируемым контроллерам, среди которых хорошо известен Arduino. Подключение к его плате производится тремя проводами. По двум подается питающее напряжение, а по третьему - управляющий сигнал. Инструкция сервопривода с цифровым управлением предусматривает наличие в контроллере простой программы, позволяющей считывать с потенциометра показания и переводить их в число. Затем оно преобразуется в команду передачи на поворот вала сервопривода в заданное положение. Программа записывается на диске, а затем передается на контроллер

\subsection{Вывод по работе}
Мы добились практически стабильной работы сервопривода, но отличающейся от теоретической идеальной модели в силу механических погрешностей, неисправностей и шумов в устройствах.
\end{document}
