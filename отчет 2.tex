\documentclass[12pt,a4paper]{scrartcl}
\usepackage[utf8]{inputenc}
\usepackage[english,russian]{babel}
\usepackage{indentfirst}
\usepackage{misccorr}
\usepackage{graphicx}
\usepackage{amsmath}
\begin{document}
\begin{titlepage}
  \begin{center}
    \large
 
    УНИВЕРСИТЕТ ИТМО
    \vspace{0.25cm}
     
    Факультет Систем управления и робототехники
     
    Кафедра мехатроники
    \vfill
 
    \textsc{Отчет по лабораторной работе №2}\\[5mm]
     
    {\LARGE Работа со сдвиговым регистром}

\end{center}
\vfill
 
\newlength{\ML}
\settowidth{\ML}{«\underline{\hspace{0.7cm}}» \underline{\hspace{2cm}}}
\hfill\begin{minipage}{0.4\textwidth}
  Руководитель: Куприянов Д.В.\\
  Выполнила: Морозова Е.Д. \\
  Студентка 1 курса группы Р3125\\
  
  
\end{minipage}%
\bigskip

 
\begin{center}
  Санкт-Петербург, 2018г.
\end{center}
\end{titlepage}

\begin{tableofcontents}
\end{tableofcontents}
    \newpage
    
\section{Введение}

\label{sec:intro}
 
% Что должно быть во введении

\subsection{Описание работы}
Познакомимся со сдвиговым регистром 74HC595 и логикой его работы при помощи микроконтроллеров Arduino Uno, Arduino Mega и программы Arduino IDE.\\
\\

\subsection{Цель работы}
Научиться работать со сдвиговым регистром, управляя его работой с помощью микроконтроллера, решить сопутствующие проблемы, добиться попеременного мигания подключенных светодиодов через регистр, понять общий принцип работы регистра при помощи Arduino Uno, Mega и Arduino IDE.

 	\newpage
 
\section{Оборудование, использованное в работе}
\begin{figure}[h!]  
	\centering
	\includegraphics[width=0.5\textwidth]{pic1.eps} %тут в конце адрес картинки 
	\caption{Микроконтроллер Arduino Uno} 
	\label{image:pic1} %% внутренняя ссылка на картинку в рамках документа 
\end{figure}	
\begin{figure}[h!]  
	\centering
	\includegraphics[width=0.5\textwidth]{pic4.eps} %тут в конце адрес картинки 
	\caption{Сдвиговый регистр} 
	\label{image:pic4} %% внутренняя ссылка на картинку в рамках документа 
\end{figure}
  \newpage
  \section{Ход работы}
  \subsection{Схема подключения}
  \begin{figure}[h!]  
  	\centering
  	\includegraphics[width=1\textwidth]{pic5.eps} %тут в конце адрес картинки 
  	\caption{Схема подключения} 
  	\label{image:pic5}
  \end{figure}
   
\subsection{Подключение к Ардуино}
  \begin{figure}[h!]  
	\centering
	\includegraphics[width=0.5\textwidth]{pic7.eps} %тут в конце адрес картинки 
	\caption{Подключение к Ардуино Уно} 
	\label{image:pic7}
\end{figure}

  \begin{figure}[h!]  
	\centering
	\includegraphics[width=0.5\textwidth]{pic8.eps} %тут в конце адрес картинки 
	\caption{Подключение к Ардуино Мега} 
	\label{image:pic8}
\end{figure}
  \newpage
  \subsection{Код работы}
 \begin{verbatim}
# include <SPI.h> 
enum { REG_SELECT = 8 }; 
void setup () 
{ 
SPI . begin () ; 
pinMode ( REG_SELECT , OUTPUT ) ; 
digitalWrite ( REG_SELECT , LOW ) ; 
SPI . transfer (0) ; 
digitalWrite ( REG_SELECT , HIGH ) ; 
} 
void rotateLeft ( uint8_t & bits ) 
{ 
uint8_t high_bit = bits & (1 « 7) ? 1 : 0; 
bits = ( bits « 1) | high_bit ; 
} 
void loop () 
{ 
static uint8_t nomad = 1; 
digitalWrite ( REG_SELECT , LOW ) ; 
SPI . transfer ( nomad ) ; 
digitalWrite ( REG_SELECT , HIGH ) ; 
rotateLeft ( nomad ) ; 
delay (1000)
}
 \end{verbatim} 
 
\section{Результат работы}

\subsection{Принцип работы}
Использование сдвигового регистра позволит получить дополнительно 8  выхода, задействуя всего 3 выхода на Arduino. Причем это не предел, добавив еще одну микросхему получим на выходе уже 16 выходов, и так добавлять микросхемы можно до бесконечности, а на Arduino будут заняты всё те же 3 выхода. Считывать данные из регистра можно одновременно из всех ячеек. Именно это его свойство помогает нам работать с кучей светодиодов.

  \begin{figure}[h!]  
	\centering
	\includegraphics[width=1\textwidth]{pic6.eps} %тут в конце адрес картинки 
	\caption{Принципиальная схема подключения} 
	\label{image:pic6}
\end{figure}

\newpage

\subsection{Вывод по работе}
Работа была выполнена, однако из-за плохо работающего регистра (сделать такой вывод нам позволило то, что мы работали с микроконтроллерами Arduino Mega и Arduino Uno, а результат был один) светодиоды мегали не стабильно, иногда не чередуясь, а горя все одновременно. Тем не менее принцип работы был ясен и понятен.

  \begin{figure}[h!]  
	\centering
	\includegraphics[width=0.5\textwidth]{pic9.eps} %тут в конце адрес картинки 
	\caption{Результат работы} 
	\label{image:pic9}
\end{figure}
\end{document}