\documentclass[12pt,a4paper]{scrartcl}
\usepackage[utf8]{inputenc}
\usepackage[english,russian]{babel}
\usepackage{indentfirst}
\usepackage{misccorr}
\usepackage{graphicx}
\usepackage{amsmath}
\begin{document}
\begin{titlepage}
  \begin{center}
    \large
 
    УНИВЕРСИТЕТ 
    \vspace{0.25cm}
     
    Факультет 
     
    Кафедра 
    \vfill
 
    \textsc{Отчет по лабораторной работе №3}\\[5mm]
     
    {\LARGE Работа с дисплеем}

\end{center}
\vfill
 
\newlength{\ML}
\settowidth{\ML}{«\underline{\hspace{0.7cm}}» \underline{\hspace{2cm}}}
\hfill\begin{minipage}{0.4\textwidth}
  Руководитель:\\
  Выполнила:  \\

  
  
\end{minipage}%
\bigskip

 
\begin{center}
  Санкт-Петербург
\end{center}
\end{titlepage}

\begin{tableofcontents}
\end{tableofcontents}
    \newpage
    
\section{Введение}

\label{sec:intro}
 
\subsection{Описание работы}
Провести лабораторную работу по работе с дисплеем и понять принцип работы библиотек при помощи микроконтроллера Arduino Uno и программы Arduino IDE.\\
\\

\subsection{Цель работы}
Научиться работать с дисплеем, управляя его работой с помощью микроконтроллера, решить сопутствующие проблемы, понять общий принцип работы с библиотека при помощи Arduino Uno и Arduino IDE.

 	\newpage
 
\section{Оборудование, использованное в работе}
\begin{figure}[h!]  
	\centering
	\includegraphics[width=0.5\textwidth]{pic1.eps} %тут в конце адрес картинки 
	\caption{Микроконтроллер Arduino Uno} 
	\label{image:pic1} %% внутренняя ссылка на картинку в рамках документа 
\end{figure}	
\begin{figure}[h!]  
	\centering
	\includegraphics[width=0.5\textwidth]{pic10.eps} %тут в конце адрес картинки 
	\caption{LCD дисплей} 
	\label{image:pic10} %% внутренняя ссылка на картинку в рамках документа 
\end{figure}
  \newpage
  \section{Ход работы}
  \subsection{Схема подключения}
  
  \begin{figure}[h!]  
  	\centering
  	\includegraphics[width=0.5\textwidth]{pic11.eps} %тут в конце адрес картинки 
  	\caption{Схема подключения дисплея к Ардуино} 
  	\label{image:pic11}
  \end{figure}

 \begin{figure}[h!]  
	\centering
	\includegraphics[width=0.5\textwidth]{pic12.eps} %тут в конце адрес картинки 
	\caption{Наша схема} 
	\label{image:pic12}
\end{figure}
  \subsection{Код работы}
 \begin{verbatim}
# include <Wire .h> 
# include < LiquidCrystal_I2C .h> 
LiquidCrystal_I2C lcd (0 x27 ,16 ,2) ;
void setup () 
{ 
lcd . init () ; 
lcd . backlight () ;
lcd . print ("hello , Anton !") ; 
lcd . setCursor (8 , 1) ; 
lcd . print ("LCD ,1602 ") ; 
} 
void loop () 
{ 
lcd . setCursor (0 , 1) ; 
lcd . print ( millis () /1000) ; 
}


 \end{verbatim}

\section{Результат работы}

\subsection{Принцип работы}
Этот монохромный дисплей имеет опциональную подсветку и может отображать 2 строки по 16 символов. Разрешение символов — 5x8 точек. Существует поддержка кириллицы.

\subsection{Вывод по работе}
Одной из задач работы было попробовать самостоятельно написать библиотеку для работы с дисплеем, однако наши попытки оказались неудачными, что показывает всю сложность ее написания. Работа с готовой библиотекой для дисплея оказалась простой и прошла удачно. 
\end{document}
