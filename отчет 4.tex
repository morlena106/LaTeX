\documentclass[12pt,a4paper]{scrartcl}
\usepackage[utf8]{inputenc}
\usepackage[english,russian]{babel}
\usepackage{indentfirst}
\usepackage{misccorr}
\usepackage{graphicx}
\usepackage{amsmath}
\begin{document}
\begin{titlepage}
  \begin{center}
    \large
 
    УНИВЕРСИТЕТ 
    \vspace{0.25cm}
     
    Факультет
     
    Кафедра 
    \vfill
 
    \textsc{Отчет по лабораторной работе №4}\\[5mm]
     
    {\LARGE Исследование энтропии}

\end{center}
\vfill
 
\newlength{\ML}
\settowidth{\ML}{«\underline{\hspace{0.7cm}}» \underline{\hspace{2cm}}}
\hfill\begin{minipage}{0.4\textwidth}
  Руководитель: \\
  Выполнила: \\
 
  
  
\end{minipage}%
\bigskip

 
\begin{center}
  Санкт-Петербург
\end{center}
\end{titlepage}

\begin{tableofcontents}
\end{tableofcontents}
    \newpage
    
\section{Введение}

\label{sec:intro}
 
\subsection{Описание работы}
Провести лабораторную работу по исследованию энтропии при помощи микроконтроллера Arduino Uno, различных датчиков и программы Arduino IDE.\\
\\

\subsection{Цель работы}
Перебрать источники энтропии, получаемые со входа различных датчиков, в нашем случае сонара. Провести серию опытов, обработать и построить графики распределения случайных чисел, полученных на монитор серийного порта. 

 	\newpage
 
\section{Оборудование, использованное в работе}
\begin{figure}[h!]  
	\centering
	\includegraphics[width=0.5\textwidth]{pic1.eps} %тут в конце адрес картинки 
	\caption{Микроконтроллер Arduino Uno} 
	\label{image:pic1} %% внутренняя ссылка на картинку в рамках документа 
\end{figure}	
\begin{figure}[h!]  
	\centering
	\includegraphics[width=0.5\textwidth]{pic13.eps} %тут в конце адрес картинки 
	\caption{Сонар} 
	\label{image:pic10} %% внутренняя ссылка на картинку в рамках документа 
\end{figure}
  \newpage
  \section{Ход работы}
  \subsection{Схема подключения}
  
  \begin{figure}[h!]  
  	\centering
  	\includegraphics[width=0.5\textwidth]{pic14.eps} %тут в конце адрес картинки 
  	\caption{Схема подключения сонара к Ардуино} 
  	\label{image:pic11}
  \end{figure}

  \subsection{Код работы}
  Программа считывания данных с сонара, графики распределения данных указаны в приложении на рис.4 и рис.5.
 \begin{verbatim}
int echoPin = 9; 
int trigPin = 8; 

void setup() { 
Serial.begin (9600); 
pinMode(trigPin, OUTPUT); 
pinMode(echoPin, INPUT); 
} 

void loop() { 
int duration, cm; 
digitalWrite(trigPin, LOW); 
delayMicroseconds(2); 
digitalWrite(trigPin, HIGH); 
delayMicroseconds(10); 
digitalWrite(trigPin, LOW); 
duration = pulseIn(echoPin, HIGH); 
cm = duration / 58;
Serial.print(cm); 
Serial.println(" cm"); 
delay(100);
}
 \end{verbatim}
 
 Программа, снимающая шумы с портов и функцией рандом в лупе, графики распределения данных указаны в приложении на рис.6 и рис.7. 
 \begin{verbatim} 
 long randNumber;
 #include <NewPing.h> //подключение библиотеки
 
 #define TRIGGER_PIN 31 //пин подключения контакта Trig
 #define ECHO_PIN 30 //Пин подключения контакта Echo
 #define MAX_DISTANCE 700 //максимально-измеряемое расстояние
 //создание объекта дальномера
 NewPing sonar(TRIGGER_PIN, ECHO_PIN, MAX_DISTANCE);
 void setup() {
 	// put your setup code here, to run once:
 	Serial.begin(9600);
 	
 }
 
 void loop() {
 	randomSeed(analogRead(0));
 	randNumber = random(1000);
 	Serial.println(randNumber);
 	delay(100);// put your main code here, to run repeatedly:
 	
 }
 \end{verbatim}
 
 Далее функция рандом была помещена в сетап, считаны шумы с портов и построены графики распределения данных, указанные в приложении на рис.8 и рис.9.
 \newpage
\section{Результат работы}

\subsection{Вывод по работе}
Нашей задачей было познакомиться с функцией рандом и ее использовании в ардуино, а так же исследовать энтропию пространства при помощи микроконтроллера ардуино и дальномера, что мы и сделали, простроив графики распределения. 

\section{Приложения}
 \begin{figure}[h!]  
	\centering
	\includegraphics[width=1\textwidth]{tabl1.eps} %тут в конце адрес картинки 
	\caption{Таблица 1} 
\end{figure}

 \begin{figure}[h!]  
	\centering
	\includegraphics[width=1\textwidth]{tabl2.eps} %тут в конце адрес картинки 
	\caption{Таблица 2} 

\end{figure}

 \begin{figure}[h!]  
	\centering
	\includegraphics[width=1\textwidth]{tabl3.eps} %тут в конце адрес картинки 
	\caption{Таблица 3} 

\end{figure}

 \begin{figure}[h!]  
	\centering
	\includegraphics[width=1\textwidth]{tabl4.eps} %тут в конце адрес картинки 
	\caption{Таблица 4} 

\end{figure}

 \begin{figure}[h!]  
	\centering
	\includegraphics[width=1\textwidth]{tabl5.eps} %тут в конце адрес картинки 
	\caption{Таблица 5} 
	
\end{figure}

 \begin{figure}[h!]  
	\centering
	\includegraphics[width=1\textwidth]{tabl6.eps} %тут в конце адрес картинки 
	\caption{Таблица 6} 

\end{figure}
\end{document}
