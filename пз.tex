 \documentclass[12pt,a4paper]{scrartcl}
 \usepackage[utf8]{inputenc}
 \usepackage[english,russian]{babel}
 \usepackage{indentfirst}
 \usepackage{misccorr}
 \usepackage{graphicx}
 \usepackage{amsmath}
 
 \newcommand{\texzadtitle}[6]{%
 	\begin{titlepage}
 		
 		\clearpage\vspace*{\fill}
 		\begin{center}
 			
 			ПОЯСНИТЕЛЬНАЯ ЗАПИСКА \\
 			ТЕХНИЧЕСКОГО ПРЕДЛОЖЕНИЯ \\
 			\vskip 2ex
 			\parbox{12cm}{\centering для #1}
 			\vskip 3em\normalfont
 		\end{center}
 		\vspace*{\fill}
 		\vfill
 		\begin{tabular}[t]{l}
 			%% Стадия разработки:~#2\vspace{1cm}\\
 			Составители:~#3\\
 			Исполнители:~#4\vspace{1cm}\\
 			%% Заказ~\hbox to 2.5cm{\hrulefill}, шифр:~#5\vspace{1cm}\\
 			Согласовано:\underline{\hspace{3cm}} \\
 		\end{tabular}
 		\vfill
 		\vfill
 		\centerline{#6}
 		\vspace*{1cm}
 		\centerline{\number\year\,г.}
 	\end{titlepage}
 	\setcounter{page}{2}
 }
 
 \makeatother
 
 % Paragraph
 
 \begin{document}
 	
 	\texzadtitle{прототипа автономной системы для предмета "Информационные технологии"}%//Здесь название технического задания
 	{разработка РКД}%%Стадия разработки
 	{}%Составитель
 	{}%Исполнитель
 	{АФАР}%Шифр
 	{}%
 	\renewcommand{\sectionmark}[1]{\markboth{#1}{}}
 		
 	\begin{tableofcontents}
 	\end{tableofcontents}
 	\newpage
 	
 	\section{Общие положения}
 		\subsection{Наименование проектируемой системы }
 		"Умная теплица", прибор-анализатор помещения, компонент умного дома.
 		
 		\subsection{Организации, участвующие в разработке }
 	Создано при поддержке Университета ИТМО, MT.Lab и магазина Roboshop.
 	
 		\subsection{Цели разработки системы }
 	Создание автономного устройства, способного анализировать различные характеристики помещения, в котором находится, помогая в хозяйстве, быту и работе.
 		
 		\subsection{Назначение и области использования системы }
 	Анализирование состояния любого помещения касательно концентрации газов, температуры, влажности и освещенности, и передача данных пользователю. 
 	
 		\subsection{Соответствие системы нормам и правилам техники безопасности }
 	Соответствует нормам и правилам техники безопасности при работе с микроконтроллерами Ардуино, компоненты и провода изолированы, подаваемое и выходное на проводах напряжение недостаточное для удара пользователя при контакте с незащищенными компонентами внутри устройста. Все электрические компоненты находятся внутри деревянного корпуса, обеспечивающего полную изоляцию и безопасность.
 	 
 	Рекомендуемое время работы от одного до нескольких часов. При более длительной эксплуатации советуется делать перерывы в работе, отключая питание, давая контроллеру и аккумулятору не перегреться. Взрывобезопасно.
 	
 		\subsection{Сведения об использованных при разработке нормативно-технических документах }
 	При подклюении датчиков и модулей и написании работающего кода для микроконтроллера использовались официальные технические документации для электронных компонентов (Datasheet), доступные в интернете.
 	
 		\subsection{Очередность создания системы}
 	Разработка идеи проекта 01 марта 2018 года - 10 марта 2018 года
 	
 	Разработка технического задания 10 марта 2018 года - 20 марта 2018 года
 	
 	Техническое проектирование 20 марта 2018 года - 20 мая 2018 года
 	
 	Рабочее проектирование 20 мая 2013 года - 24 июня 2018 года
 	
 	Приёмо-сдаточные испытания  25 июня 2018 года
 	
 	\section{Описание процесса деятельности}
 	Закупка нужных компонентов, поиск и изучение тезхнической документации для каждого компонента, проверка работы кажого датчика с написанием соответствующего кода, сборка цельного прибора и написание рабочего кода, подключение Блютуз модуля и разработка вывода данных на телефон, моделирование и создание из фанеры корпуса. 
 	
 	\section{Основные технические решения}
 		\subsection{Решения по структуре системы }
 	Микроконтроллер Arduino Uno, система различных датчиков и Bluetooth-модуль для связи с телефоном.	
 	
 		\subsection{Средства и способы взаимодействия для информационного обмена между компонентами Системы }
 	Взаимодействующие компоненты внутри системы (Микроконтррллер и датчики, микроконтроллер и  Bluetooth-модуль) и взаимодействие с внешними информационными устройствами ( Bluetooth-модуль и внешние устройства совместимые с Bluetooth)
 	
 		\subsection{Диагностирование прикладных программных средств }
 	Специальных процедур диагностирования состояния компонентов системы не предусмотрено.
 	
 		\subsection{Обеспечение заданных в техническом задании характеристик, определяющих качество Системы }
 			\subsubsection{Надежность}
 	Входное напряжение в 5В и полная изоляция благодаря цельному корпусу и фанеры обеспечивают безопасность эксплуатациии пользователем.
 			\subsubsection{Удобство применения}
 	Размер собранного устройства составляем 170х90х65 мм, что обеспечивает удобность при переноске, установке и использовании. Можно переносить в сумке и использовать при надобности или установить в помещении. Работает от 6 батареек АА в аккумуляторе. Корпус вырезан из обработанной фанеры.
 			\subsubsection{Функциональность}
 	Команды для отправки на устройство через приложения Bluetooth Терминала очень просты, в зависимости от надобности пользователя, могут выводит все характеристики или одну определенную на расстоянии до 80 м от прибора.
 	
 		\subsection{Состав функций, комплексов задач, реализуемых системой }
 	Сбор данных о состоянии помещения и передача полученных данных на любое устройство, поддерживающее Bluetooth.
 	
 		\subsection{Комплекс технических средств и его размещение на объекте автоматизации}
 	
 	*Датчик температуры и влажности DHT11
 	
 	*Датчик газов MQ-2
 	
 	*Пироэлетрияечкий инфакрасный датчик движения PIR
 	
 	*Датчик интенсивности света
 	
 	*Bluetooth модуль
 	
 	Для удобства размещены на макетной плате внутри корпуса, под которой находится аккумулятор для питания всего устройства.
 	
 		\subsection{Объем, состав, способы организации, последовательность обработки информации}
 	Через приложения Bluetooth Терминала сообщение с командой через Bluetooth модуль передается и обрабатывается на Ардуино Уно, снимающего показания и с датчиков и возвращающего данные на устройство.
 	
 	Сообщение с командой состоит всего из одной буквы для удобства пользователя:
 	r - room (данные со всех датчиков), g - gas, t - temperature and humidity, l-light, m-move.
 	
 	Ответное сообщение содержит лишь запрашиваемые данные в единицах измерения.
 	
 	Время затрачиваемое на запрос и получение ответных данных составляет 0,1 секунды.
 		\subsection{Состав программных продуктов, языки деятельности, алгоритмы процедур и операций и методы их реализации}
 	Используемый язык разработки Arduino C; программа, обеспечивающая компиляцию и перенос кода на плату микроконтроллера Adduino IDE; Микроконтроллер Ардуино, работающий на чипе  ATmega 328; приложение Bluetooth Терминал для передачи данным по каналу  Bluetooth. 
 	
 	В будущем планируется заменить  Bluetooth на Wi-fi и написать собственное приложение для связи по Wi-fi.
 	\section{Мероприятия по подготовке объекта автоматизации к вводу системы в действие}
 	Обучение работы на лазерном фрезере для создания корпуса.
 	
 	Скачивание на телефонное устройство приложения связи по каналу  Bluetooth и обмену данными.
 	
 	Замена батареек в аккумуляторе. 
 	
 	Проверка каждого датчика в отдельности перед презентацией устройства.
 	
 		\vfill
 	\texzadfooter{
 		{Выполнили:\,}%
 		
 		{Утвердил:\,}%
 		
 		{Подпись:\,\underline{\hspace{3cm}}}%
 	 \end{document}
