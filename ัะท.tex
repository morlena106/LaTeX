 \documentclass[12pt,a4paper]{scrartcl}
 \usepackage[utf8]{inputenc}
 \usepackage[english,russian]{babel}
 \usepackage{indentfirst}
 \usepackage{misccorr}
 \usepackage{graphicx}
 \usepackage{amsmath}
 
 \newcommand{\texzadtitle}[6]{%
 	\begin{titlepage}
 		
 		\clearpage\vspace*{\fill}
 		\begin{center}
 			
 			ТЕХНИЧЕСКОЕ ЗАДАНИЕ \\
 			\vskip 2ex
 			\parbox{12cm}{\centering на разработку #1}
 			\vskip 3em\normalfont
 		\end{center}
 		\vspace*{\fill}
 		\vfill
 		\begin{tabular}[t]{l}
 			%% Стадия разработки:~#2\vspace{1cm}\\
 			Составители:~#3\\
 			Исполнители:~#4\vspace{1cm}\\
 			%% Заказ~\hbox to 2.5cm{\hrulefill}, шифр:~#5\vspace{1cm}\\
 			Согласовано:\underline{\hspace{3cm}} \\
 		\end{tabular}
 		\vfill
 		\vfill
 		\centerline{#6}
 		\vspace*{1cm}
 		\centerline{\number\year\,г.}
 	\end{titlepage}
 	\setcounter{page}{2}
 }
 
 \makeatother
 
 % Paragraph
 
 \begin{document}
 	
 	\texzadtitle{прототипа автономной системы для предмета "Информационные технологии"}%//Здесь название технического задания
 	{разработка РКД}%%Стадия разработки
 	{}%Составитель
 	{}%Исполнитель
 	{АФАР}%Шифр
 	{}%
 	\renewcommand{\sectionmark}[1]{\markboth{#1}{}}
 	
 	\begin{tableofcontents}
 	\end{tableofcontents}
 	\newpage

 	\section{Наименование и сроки выполнения работы}
 	\label{sec:purpose}
 	
 	\subsection{Наименование работы:}
 	Разработка и изготовление прибора-анализатора помещения.
 	
 	\subsection{Сроки выполнения.}
 	Март 2018 - июнь 2018
 	
 	\section{Цель выполнения работы. Наименование
 		изделия.}
 	\label{sec:requirements}
 
 	\subsection{Цель выполнения работы:}
 	Создание прибора-компонента для умного дома
 	
 	\subsection{Задачи,решение которых обеспечивает достижение поставленной цели:}
 	{разработка электрической схемы подключения датчиков к плате Arduino}
 	
 	{написание кода работы датчиков}
 	
 	{изготовление корпуса}
 	
 	{расписание системы для вывода данных пользователю}
 	
 	{тестирование устройства в условиях эксплуатации}
 	
 	{презентация прототипа }
 	
 	\section{Назначение прибора}
 	Анализ состояния помещения и передача данных пользователю
 	\section{Компоненты}
 	Микроконтроллер arduino uno
 	
 	Датчик температуры и влажности DHT11
 	
 	Датчик газов MQ-2
 	
 	Пироэлетрияечкий инфакрасный датчик движения PIR
 	
 	Датчик интенсивности света
 	
 	Провода папа-мама
 	
 	Резисторы
 	
 	Макетная плата
 	
 	Bluetooth модуль
 	
 	Аккумулятор питания
 	\section{Технические характеристики разрабатываемого прототипа}
 	
 	Рабочее напряжение 5В
 	
 	Предельное входное напряжение 6-20В
 	
 	Рекомендуемое входное напряжение 7-12В
 	
 	Измерение влажности: 20-80 +/-5 процентов
 	
 	Температуры 0-50*С +/-2 процента
 	
 	Измерение с максимальной частотой одно измерение в секунду
 	
 	Диапазон измерений концентрации газов:
 	
 	* Пропан: 200–5000 ppm
 	
 	* Бутан: 300–5000 ppm
 	
 	* Метан: 500–20000 ppm
 	
 	* Водород: 300–5000 ppm
 	
 	* Диапазон чувствительности: до 6 метров. Угол обзора 110° x 70°;
 	
 	* Чувствительность: 65536 градаций и измерения интенсивности света с точностью до 1 люкса
 	Передача данных на расстоянии до 80м
 	\section{Требования}
 	\subsection{Требования к прототипу:}
 	Анализ помещения на концентрацию газов, таких как изобутан, пропан, метан, водород, присутствие алкогольных паров и табачного дыма, температуру и влажность, наличие движения в комнате и освещённость.
 	
 	\subsection{Требования к эксплуатации:}
 	Отсутствие механического воздействия и механических повреждений, устанавливать и хранить в дали от детей до 6 лет, диапазон рабочей температуры крайне высок и не требует особых климатических условий. Рабочее напряжение 5-20В, рекомендуемое 7-12В. 
 	
 	\subsection{Требования к ремонту:}
 	Аккуратно открыть верхнюю крышку корпуса и проверить соединение проводов и датчиков, если проблема не решена, то заменить датчики или батарейки в аккумуляторе. 
 	
 	\section{Порядок контроля}
 	\subsection{Виды испытаний:}
 	Предварительные испытания, опытная эксплуатация 
 	
  	\subsubsection{Проведение предварительных испытаний.}
 	
 	Фиксирование выявленных неполадок в Протоколе испытаний.
 	
 	Устранение выявленных неполадок.
 	
 	Проверка устранения выявленных неполадок.
 	\subsection{Место проведения: } Университет ИТМО
 	
 	\subsection{Состав проверочной комиссии:}
 	Куприянов Д.В.
 	\section{Требования к документации }
 	Рабочими документами являются документация комплектующих
 	
 	компонентов и всего прототипа согласно стандартам и требованиям ЕСКД.
 	
 	\section{Источники разработки }
 	Вдохновение, посланное преподавателем на одной из пар во время огласки о работе над будущими проектами и временной интервал разработки идеи в один час.
 	
 	% \clearpage
 	\vfill
 	\texzadfooter{
 		{Выполнили:\,}%
 		
 		{Утвердил:\,}%
 		
 		{Подпись:\,\underline{\hspace{3cm}}}%
 	}
 	
 \end{document}
